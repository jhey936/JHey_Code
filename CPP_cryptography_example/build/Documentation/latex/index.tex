A simple command line tool for encrypting/decrypting text using classical ciphers

\section*{Building {\ttfamily mpags-\/cipher}}

Compilation of {\ttfamily mpags-\/cipher} requires the \href{http://www.cmake.org}{\tt C\+Make} build tool, plus a C++11 compatible compiler (G\+CC 4.\+8 or better, Clang 3.\+4 or better are recommended) and {\ttfamily make} on a U\+N\+IX operating system. Windows platforms with Visual Studio 2015 or better are also expected to work, but not tested.

To build from a clone of this repository, open a terminal window and change directory into that holding this R\+E\+A\+D\+ME. Create a build directory in which to run {\ttfamily cmake} and the build, and change into it\+:


\begin{DoxyCode}
1 $ ls
2 CMakeLists.txt   LICENSE          MPAGSCipher      README.md        mpags-cipher.cpp
3 $ mkdir ../Build
4 $ cd ../Build
\end{DoxyCode}


Run {\ttfamily cmake} in this directory, pointing it to the directory holding this R\+E\+A\+D\+ME, and consequently the top level C\+Make script for the project\+:


\begin{DoxyCode}
1 $ cmake ../<this dir>
2 ... system specific output ...
3 -- Configuring done
4 -- Generating done
5 -- Build files have been written to: ... your build dir path ...
6 $
\end{DoxyCode}


The exact output will depend on your system, compiler and build directory location, but you should not see any errors. C\+Make will generate Makefiles by default on U\+N\+IX platforms, so to build, simply run {\ttfamily make} in the build directory\+:


\begin{DoxyCode}
1 $ ls
2 CMakeCache.txt      CMakeFiles          Makefile            cmake\_install.cmake
3 $ make
4 ... verbose output ...
5 [100%] Built target mpags-cipher
6 ...
7 $
\end{DoxyCode}


Again, the exact output will be system specific, but you should see the {\ttfamily mpags-\/cipher} target built without error. The resulting {\ttfamily mpags-\/cipher} executable can then be run directly, and provides the following command line options\+:


\begin{DoxyCode}
1 $ ./mpags-cipher --help
2 Usage: mpags-cipher [-i/--infile <file>] [-o/--outfile <file>] [-c/--cipher <cipher>] [-k/--key <key>]
       [--encrypt/--decrypt]
3 
4 Encrypts/Decrypts input alphanumeric text using classical ciphers
5 
6 Available options:
7 
8   -h|--help
9                       Print this help message and exit
10 
11   -v|--version
12                       Print version information
13 
14   -i|--infile FILE
15                       Read text to be processed from FILE
16                       Stdin will be used if not supplied
17 
18   -o|--outfile FILE
19                       Write processed text to FILE
20                       Stdout will be used if not supplied
21 
22   -c|--cipher CIPHER
23                       Specify the cipher to be used to perform the encryption/decryption
24                       CIPHER can be caesar, playfair or vigenere - caesar is the default
25 
26   -k|--key KEY
27                       Specify the cipher KEY
28                       A null key, i.e. no encryption, is used if not supplied
29 
30   --encrypt
31                       Will use the cipher to encrypt the input text (default behaviour)
32 
33   --decrypt
34                       Will use the cipher to decrypt the input text
\end{DoxyCode}


If no input file is supplied, {\ttfamily mpags-\/cipher} will wait for user input from the keyboard until R\+E\+T\+U\+RN followed by C\+T\+R\+L-\/D are pressed. It will then echo the input to stdout or write it to the file supplied with the {\ttfamily -\/o} option.

To ensure the input text can be used with the character sets known to classical ciphers, it is transliterated using the following rules\+:


\begin{DoxyItemize}
\item Alphabetical characters are converted to uppercase
\item Digits are translated to their English equivalent words (e.\+g. \textquotesingle{}0\textquotesingle{} -\/$>$ \char`\"{}\+Z\+E\+R\+O\char`\"{})
\item All other characters (punctuation) are discarded
\end{DoxyItemize}

At present, the Caesar, Playfair and Vigenere ciphers are supported.

\section*{Testing}

After building the M\+P\+A\+G\+S\+Cipher library it can be tested by running {\ttfamily ctest -\/\+VV} from the build directory.

\section*{Source Code Layout}

Under this directory, the code and associated files are organised as follows\+:


\begin{DoxyCode}
1 MPAGS-Code
2 ├── README.md             This file, describes the project
3 ├── LICENSE               License file, in our case MIT
4 ├── CMakeLists.txt        CMake build script
5 ├── mpags-cipher.cpp      Main program C++ source file
6 ├── Documentation         Subdirectory for documentation of the MPAGSCipher library
7 │   ├── CMakeLists.txt
8 │   └── Doxyfile.in
9 ├── MPAGSCipher           Subdirectory for MPAGSCipher library code
10 │   ├── CMakeLists.txt
11 │   ├── Alphabet.hpp
12 │   ├── CaesarCipher.cpp
13 │   ├── CaesarCipher.hpp
14 │   ├── Cipher.hpp
15 │   ├── CipherFactory.cpp
16 │   ├── CipherFactory.hpp
17 │   ├── CipherMode.hpp
18 │   ├── CipherType.hpp
19 │   ├── PlayfairCipher.cpp
20 │   ├── PlayfairCipher.hpp
21 │   ├── ProcessCommandLine.cpp
22 │   ├── ProcessCommandLine.hpp
23 │   ├── TransformChar.cpp
24 │   ├── TransformChar.hpp
25 │   ├── VigenereCipher.cpp
26 │   └── VigenereCipher.hpp
27 ├── Testing               Subdirectory for testing the MPAGSCipher library
28 │   ├── CMakeLists.txt
29 │   ├── catch.hpp
30 │   ├── testCaesarCipher.cpp
31 │   ├── testCatch.cpp
32 │   ├── testCiphers.cpp
33 │   ├── testHello.cpp
34 │   ├── testPlayfairCipher.cpp
35 │   ├── testProcessCommandLine.cpp
36 │   ├── testTransformChar.cpp
37 │   └── testVigenereCipher.cpp
\end{DoxyCode}


\section*{Copying}

{\ttfamily mpags-\/cipher} is licensed under the terms of the M\+IT License. Please see the file \mbox{[}{\ttfamily L\+I\+C\+E\+N\+SE}\mbox{]}(L\+I\+C\+E\+N\+SE) for full details. 